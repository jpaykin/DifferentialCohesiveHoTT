% compile with e.g.:
% latexmk -pdflatex='xelatex %O %S' -pvc -pdf slides-oxford-1.tex
\documentclass{beamer}
% %\usepackage[lmargin=2.5cm,rmargin=2.5cm,tmargin=2.5cm,bmargin=2.5cm]{geometry}
\usepackage{fullpage}
\usepackage{amsmath}
\usepackage{amssymb}
\usepackage{amsthm}
\usepackage{phonetic} % for esh
\usepackage{enumerate}

\usepackage{tikz}
\usetikzlibrary{arrows}

\usepackage{authblk} % For authors' affiliations
\makeatletter
\renewcommand\AB@affilsepx{, \protect\Affilfont}
\makeatother


\usepackage[hidelinks]{hyperref}
\usepackage{natbib}

\usepackage[capitalize,noabbrev]{cleveref} % must be loaded after hyperref

\usepackage{xspace}

\usepackage{wrapfig}
\usepackage{subcaption}

% Modalities
\newcommand{\Red}{\Re}
\newcommand{\Cored}{\Im}
\newcommand{\Wat}{\&}
\newcommand{\shape}{\ensuremath{\mathord{\raisebox{0.5pt}{\text{\rm\esh}}}}}
\newcommand{\submodality}{\rotatebox[origin=c]{90}{$\subset$}}


% Latin  Abbr
\newcommand{\etal}{\emph{et al.}\xspace}
\newcommand{\eg}{\emph{e.g.,}\xspace}
\newcommand{\ie}{\emph{i.e.,}\xspace}
\newcommand{\etc}{\emph{etc.}\xspace}

%%%%%%%%%%%%%%
%% Comments %%
%%%%%%%%%%%%%%
\usepackage{draft}

\newnote[Jacob]{JAG}{blue}
\newnote[Max]{MN}{blue}
\newnote[Jennifer]{JP}{blue}
\newnote[Mitchell]{MMR}{blue}
\newnote[Felix]{FW}{blue}
\newnote[Dan]{DL}{blue}
\newnote[Mike]{MS}{blue}

\newnote{fixme}{red}
\newnote{note}{red}
\newnote{todo}{red}
\newnote[Cite:]{tocite}{red}




\usepackage{fullpage}
\usepackage{amsmath}
\usepackage{amssymb}
\usepackage{amsthm}
\usepackage{phonetic} % for esh
\usepackage{enumerate}

\usepackage{tikz}
\usetikzlibrary{arrows}

\usepackage{authblk} % For authors' affiliations
\makeatletter
\renewcommand\AB@affilsepx{, \protect\Affilfont}
\makeatother

% Modalities
\newcommand{\Red}{\Re}
\newcommand{\Cored}{\Im}
\newcommand{\Wat}{\&}
\newcommand{\shape}{\ensuremath{\mathord{\raisebox{0.5pt}{\text{\rm\esh}}}}}
\newcommand{\submodality}{\rotatebox[origin=c]{90}{$\subset$}}


% Latin  Abbr
\newcommand{\etal}{\emph{et al.}\xspace}
\newcommand{\eg}{\emph{e.g.,}\xspace}
\newcommand{\ie}{\emph{i.e.,}\xspace}
\newcommand{\etc}{\emph{etc.}\xspace}



\newcommand{\ignore}[1]{}
\newcommand{\op}{\mathrm{op}} 
\newcommand{\R}{\mathbb{R}}
\newcommand{\nat}{\mathbb{N}}

%%%%%%%%%%%%%%
%% Comments %%
%%%%%%%%%%%%%%
\usepackage{draft}

\usepackage{hyperref}
\usepackage{natbib}
\usepackage[capitalize,noabbrev]{cleveref} % must be loaded after hyperref


\newnote[Jacob]{JAG}{blue}
\newnote[Max]{MN}{blue}
\newnote[Jennifer]{JP}{blue}
\newnote[Mitchell]{MMR}{blue}
\newnote[Felix]{FW}{blue}
\newnote[Dan]{DL}{blue}
\newnote[Mike]{MS}{blue}

\newnote{fixme}{red}
\newnote{note1}{red}
\newnote{todo}{red}
\newnote[Cite:]{tocite}{red}


\title{Differential Cohesive Type Theory \thanks{This
    material is based upon work supported by the National Science Foundation
    under Grant Number DMS 1321794.}}
\author{Jacob A.\,Gross}
\affil[1]{University of Pittsburgh}
\author[2]{Daniel R.\,Licata}
\affil[2]{Wesleyan University}
\author[3]{Max S.\, New}
\affil[3]{Northeastern University}
\author[4]{Jennifer Paykin}
\affil[4]{University of Pennsylvania}
\author[2]{Mitchell Riley}
\author[5]{Michael Shulman}
\affil[5]{University of San Diego}
\author[6]{Felix Wellen}
\affil[6]{Karlsruhe Institute of Technology}
\date{}


\begin{document}

% title page is not working...
\begin{frame}
  \titlepage
\end{frame}

% before this frame, draw a picture
% indicating the interpretation of HoTT in different toposes
% and explain, which of them are interesting for the talk and why
\begin{frame}
  \frametitle{HoTT as a language for $\infty$-toposes}
  We are especially interested in the topos $\mathrm{FSGrp}$ of
  formal smooth $\infty$-groupoids. \\
  $\mathrm{FSGrp}$ is given as the topos of sheaves on 
  \[ \{\mathcal C^\op(\R^n)\otimes_\R (\R\oplus V)\mid \text{ $n\in\nat$ and $V$ is nilpotent}\}^\op \]
  with a topology respecting the smooth structure on the $\R^n$s and ignoring the infinitesimals.
\end{frame}

\begin{frame}
  \frametitle{Differential Cohesive Toposes}
    \begin{tikzpicture}[node distance=2cm]
      \node (R) {$\Red$};
      \node[right of=R] (I) {$\Cored$};
      \node[right of=I] (E) {$\Wat$};
      \node[below of=I, node distance=1.6cm] (S) {$\shape$};
      \node[below of=E, node distance=1.6cm] (b) {$\flat$};
      \node[right of=b] (s) {$\sharp$};
      
      \path (R) to node {$\dashv$} (I);
      \path (I) to node {$\dashv$} (E);
      \path (S) to node {$\dashv$} (b);
      \path (b) to node {$\dashv$} (s);
      
      \path (I) to node {$\submodality$} (S);
      \path (E) to node {$\submodality$} (b);
    \end{tikzpicture}
  $\Cored$, $\shape$ and $\sharp$ are reflections. \\
  $\Red$, $\Wat$ and $\flat$ are coreflections.
\end{frame}
% explain easy modalities like shape and flat on the whiteboard
% mention the infinitesimal sierpinski topos 
% maybe mention SDG and its connection to the differential triple

\begin{frame}
  \frametitle{Differential Cohesive Type Theory}
  Easy: The reflections can be added as modalities given by axioms (HoTT-Book 7.7). \\
  % mention that working with \Cored alone already leads to new theorems in mathematics
  % mention the theorem from Mike's real cohesion
  This is not possible for the coreflections -- for them, the rules have to be changed.
  % mention Mike's real cohesion
  % stress that it is the goal of the group work to have a similar thing for differential cohesion -- with or without localization
\end{frame}

\end{document}