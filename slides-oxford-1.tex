% compile with e.g.:
% latexmk -pdflatex='xelatex %O %S' -pvc -pdf slides-oxford-1.tex
\documentclass{beamer}

\usepackage{amsmath}
\usepackage{amssymb}
\usepackage{amsthm}
\usepackage{phonetic} % for esh
\usepackage{enumerate}

\usepackage{tikz}
\usetikzlibrary{arrows}

% Modalities
\newcommand{\Red}{\Re}
\newcommand{\Cored}{\Im}
\newcommand{\Wat}{\&}
\newcommand{\shape}{\ensuremath{\mathord{\raisebox{0.5pt}{\text{\rm\esh}}}}}
\newcommand{\submodality}{\rotatebox[origin=c]{90}{$\subset$}}


% Latin  Abbr
\newcommand{\etal}{\emph{et al.}\xspace}
\newcommand{\eg}{\emph{e.g.,}\xspace}
\newcommand{\ie}{\emph{i.e.,}\xspace}
\newcommand{\etc}{\emph{etc.}\xspace}

\newcommand{\ignore}[1]{}
\newcommand{\op}{\mathrm{op}} 
\newcommand{\R}{\mathbb{R}}
\newcommand{\D}{\mathbb{D}}
\newcommand{\nat}{\mathbb{N}}

%%%%%%%%%%%%%%
%% Comments %%
%%%%%%%%%%%%%%
\usepackage{draft}


\newnote[Jacob]{JAG}{blue}
\newnote[Max]{MN}{blue}
\newnote[Jennifer]{JP}{blue}
\newnote[Mitchell]{MMR}{blue}
\newnote[Felix]{FW}{blue}
\newnote[Dan]{DL}{blue}
\newnote[Mike]{MS}{blue}

\newnote{fixme}{red}
\newnote{note1}{red}
\newnote{todo}{red}
\newnote[Cite:]{tocite}{red}


\title{Differential Cohesive Type Theory \thanks{This
    material is based upon work supported by the National Science Foundation
    under Grant Number DMS 1321794.}}
\institute{
  \inst{1} University of Pittsburgh \qquad
  \inst{2} Wesleyan University \and
  \inst{3} Northeastern University \qquad
  \inst{4} University of Pennsylvania \and
  \inst{5} University of San Diego \qquad
  \inst{6} Karlsruhe Institute of Technology
}
\author[shortname]{
  Jacob A.\,Gross \inst{1} \and 
  Daniel R.\,Licata \inst{2} \and 
  Max S.\, New\inst{3} \and
  Jennifer Paykin\inst{4} \and
  Mitchell Riley\inst{2} \and
  Michael Shulman\inst{5} \\ \and
  Felix Wellen\inst{6}
}
\date{}

\begin{document}

\begin{frame}
  \titlepage
\end{frame}


% before this frame, draw a picture
% indicating the interpretation of HoTT in different toposes
% and explain, which of them are interesting for the talk and why
% make a distinction between differential, cohesive and differential cohesive toposes
\begin{frame}
  \frametitle{Differential Cohesive Toposes}
  \begin{center}
    \begin{tikzpicture}[node distance=2cm]
      \node (R) {$\Red$};
      \node[right of=R] (I) {$\Cored$};
      \node[right of=I] (E) {$\Wat$};
      \node[below of=I, node distance=1.6cm] (S) {$\shape$};
      \node[below of=E, node distance=1.6cm] (b) {$\flat$};
      \node[right of=b] (s) {$\sharp$};
      
      \path (R) to node {$\dashv$} (I);
      \path (I) to node {$\dashv$} (E);
      \path (S) to node {$\dashv$} (b);
      \path (b) to node {$\dashv$} (s);
      
      \path (I) to node {$\submodality$} (S);
      \path (E) to node {$\submodality$} (b);
    \end{tikzpicture}
  \end{center}
  $\Cored$, $\shape$ and $\sharp$ are reflections. \\
  $\Red$, $\Wat$ and $\flat$ are coreflections. \\
  $\shape$ and $\Red$ preserve finite products.
\end{frame}
% explain easy modalities like shape and flat on the whiteboard
% mention the infinitesimal sierpinski topos 
% maybe mention SDG and its connection to the differential triple

\begin{frame}
  \frametitle{HoTT as a tool for higher differential geometry}
  There is an embedding:
  \[ \mathcal C^\infty\colon (\text{smooth manifolds})^\op \hookrightarrow \text{$\R$-algebras}\]
  \pause
  We extends the image $\mathcal C^\infty(\{\R^n\mid\text{ $n\in\nat$ }\})$ \\
  to get the site of \emph{formal cartesian spaces}
  \[ \{\underbrace{\mathcal C^\op(\R^n)\otimes_\R (\R\oplus V)}_{=:(\R^n\times\D)^\op}\mid \text{$n\in\nat$ and $V$ nilpotent, $\dim V<\infty$}\}^\op \]
  
  with a topology respecting the smooth structure on the $\R^n$s and ignoring the infinitesimals. \\
  \pause
  The sheaves on this site are called \emph{formal smooth $\infty$-groupoids}.
  \pause
  On representables $\Red$ is given by reduction of $\R$-algebras:
  \[ \Red(\R^n\times\D)=\R^n \]
\end{frame}
% explain the differential triple induced by this reduction
% and comment on its history and importance in pure mathematics

\begin{frame}
  \frametitle{Differential Cohesive Type Theory}
  Easy: The reflections can be added as modalities given by axioms (HoTT-Book 7.7). \\
  % mention that working with \Cored alone already leads to new theorems in mathematics
  % mention the no-go-theorem from Mike's real cohesion
  This is not possible for the coreflections -- for them, the rules have to be changed.
  % mention Mike's real cohesion
  % stress that it is the goal of the group work to have a similar thing for differential cohesion -- with or without localization
\end{frame}

\end{document}