\documentclass{article}
%\usepackage[lmargin=2.5cm,rmargin=2.5cm,tmargin=2.5cm,bmargin=2.5cm]{geometry}
\usepackage{fullpage}
\usepackage{amsmath}
\usepackage{amssymb}
\usepackage{amsthm}
\usepackage{phonetic} % for esh
\usepackage{enumerate}

\usepackage{tikz}
\usetikzlibrary{arrows}

\usepackage{authblk} % For authors' affiliations
\makeatletter
\renewcommand\AB@affilsepx{, \protect\Affilfont}
\makeatother


\usepackage[hidelinks]{hyperref}
\usepackage{natbib}

\usepackage[capitalize,noabbrev]{cleveref} % must be loaded after hyperref

\usepackage{xspace}

% Modalities
\newcommand{\Red}{\Re}
\newcommand{\Cored}{\Im}
\newcommand{\Wat}{\&}
\newcommand{\shape}{\ensuremath{\mathord{\raisebox{0.5pt}{\text{\rm\esh}}}}}
\newcommand{\submodality}{\rotatebox[origin=c]{90}{$\subset$}}


% Latin  Abbr
\newcommand{\etal}{\emph{et al.}\xspace}
\newcommand{\eg}{\emph{e.g.,}\xspace}
\newcommand{\ie}{\emph{i.e.,}\xspace}
\newcommand{\etc}{\emph{etc.}\xspace}

%%%%%%%%%%%%%%
%% Comments %%
%%%%%%%%%%%%%%
\usepackage{draft}

\newnote[Jacob]{JAG}{blue}
\newnote[Max]{MN}{blue}
\newnote[Jennifer]{JP}{blue}
\newnote[Mitchell]{MMR}{blue}
\newnote[Felix]{FW}{blue}
\newnote[Dan]{DL}{blue}
\newnote[Mike]{MS}{blue}

\newnote{fixme}{red}
\newnote{note}{red}
\newnote{todo}{red}
\newnote[Cite:]{tocite}{red}




% use \draftfalse to turn off comments
% use \drafttrue to turn on comments (default)
\draftfalse

\title{Differential Cohesive Type Theory\thanks{This
    material is based upon work supported by the National Science Foundation
    under Grant Number DMS 1321794.}}
\author{Jacob A.\,Gross}
\affil[1]{University of Pittsburgh}
\author[2]{Daniel R.\,Licata}
\affil[2]{Wesleyan University}
\author[3]{Max S.\, New}
\affil[3]{Northeastern University}
\author[4]{Jennifer Paykin}
\affil[4]{University of Pennsylvania}
\author[2]{Mitchell Riley}
\author[5]{Michael Shulman}
\affil[5]{University of San Diego}
\author[6]{Felix Wellen}
\affil[6]{Karlsruhe Institute of Technology}
\date{}

%\author{Jacob A.\,Gross \and Max S.\, New \and Jennifer Paykin \and Mitchell Riley
%  \and Felix Wellen \and Daniel R.\,Licata \and Michael Shulman}

\begin{document}
\maketitle

As internal languages of toposes, type theories allow mathematicians to
reason \emph{synthetically} about mathematical structures in a concise,
natural, and computer-checkable way.  
The basic system of homotopy type theory provides an internal language for
$(\infty,1)$-toposes, and has enabled a significant line of work on
synthetic homotopy theory. 
By introducing modalities into homotopy type theory,
a synthetic treatment of geometric and topological aspects 
of objects of interest in pure mathematics seems to be possible. 

We aim at constructing the rules for an extension of homotopy type theory
providing all six operations of what are called differential cohesive $(\infty,1)$-toposes.
Half of the six operations of this structure are comonadic and there is no hope of 
postulating them by adding axioms to the type theory.
With a mode theory in adjoint type theory, 
we give a basis for a solution with nice specialized rules
and five different sorts of context.
The rules of this differential
cohesive type theory manipulate these contexts in subtle ways, which express
the relationships between the modalities.

In work in progress, we are extending this type theory to dependent types, which
is not straightforward because of the dependency structure of the differential
cohesive modalities. 

\end{document}
